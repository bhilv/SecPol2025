\chapter{Software Compliance}
% Set Pagestyle to Fancy
\pagestyle{fancy}

% Clear all header and footer fields
\fancyhf{}

% Page number only in the center of the footer
\fancyfoot[C]{\thepage}

% Remove header line
\renewcommand{\headrulewidth}{0pt}
\renewcommand{\footrulewidth}{0pt}

%ONLY EDIT BELOW, REMOVE LIPSUM (PLACEHOLDER TEXT) BEFORE EDITING
\secton{Purpose}

The purpose of this policy is to ensure all Honda Group entities maintain full compliance with software licensing agreements, copyright laws, and internal software usage guidelines. This policy helps reduce legal and financial risks, promotes cybersecurity hygiene, and upholds Honda’s commitment to ethical and lawful 
business practices. 

\section{Scope}

This policy applies to:

All Honda employees, contractors, and vendors with access to Honda-owned systems.

All software used on Honda devices (desktops, laptops, servers, mobile devices, embedded systems, etc.).

All Honda business units, including R&D, manufacturing, sales, marketing, logistics, and administration.

On-premises and cloud-based software, open-source components, and SaaS platforms.

\section{Policy Statement}

Honda strictly prohibits the use of unlicensed, unauthorized, or pirated software. All software installations and use must be pre-approved, centrally managed, and tracked by the IT department or designated software asset management (SAM) teams. Violations of this policy may result in disciplinary action, up to and including termination or legal prosecution.

\section{Software Usage Requirements}

Procurement: All software must be acquired through approved procurement channels. No employee may download, install, or purchase software independently for work use.

Licensing: All software must be properly licensed. Proof of purchase, license keys, or contracts must be maintained for audit purposes.

Installation: Only authorized IT personnel may install software on Honda devices unless approved self-service portals are available.

Open-Source Software (OSS): OSS will be used only if:

    It is approved by the IT Governance team.

    Its licensing terms (e.g., GPL, MIT, Apache) are compatible with Honda’s intended use.

    It does not introduce legal, operational, or security risks.

Cloud and SaaS Applications: Usage of SaaS tools must be reviewed for data protection, compliance (e.g., GDPR, CCPA), and contractual obligations.

Prohibited Software:

Pirated, cracked, or unauthorized software.

Software used for cryptocurrency mining.

Personal-use entertainment or file-sharing apps (unless explicitly approved.

\section{Roles and Responsibilities}

Executive Leadership
    Provide funding and strategic support for software asset compliance.

IT Governance and Compliance
    Enforce the policy, conduct audits, and manage the software inventory.
    Maintain a software license register and ensure contractual compliance.

Business Units
    Request approved software via official channels.
    Report non-compliant software or licensing concerns.

Employees
    Use only approved and licensed software.
    Refrain from altering or bypassing license restrictions.
    Promptly report suspected software misuse.

\section{Software Asset Management (Sam)}
Honda will implement a formal SAM program, which includes:

    Centralized inventory of all software assets.
    Periodic audits to detect unlicensed or unused software.
    Lifecycle management (procurement to retirement).
    Vendor contract review and license renewal tracking.
    
\section{Monitor and Auditing}

Honda reserves the right to monitor devices for unauthorized software.

Internal audits will be performed annually or as needed.

Third-party audits will be conducted to ensure regulatory and contractual compliance.

\section{Violation and Disciplinary Action}

Any employee discovered using unauthorized software or violating licensing terms will face disciplinary action, which will include:
    
    Formal warnings
    Removal of system access
    Termination of employment
    Legal action or liability for damages

\section{Exceptions}

Requests for exception must be formally submitted to the IT Governance & Compliance team, accompanied by business justification and risk assessment. Approved exceptions will be documented and reviewed annually.

\section{Policy Review and Maintenance}

This policy will be reviewed annually by the Global IT Governance & Compliance team or upon significant software-related incidents or regulatory changes.