\chapter{Disaster Recovery Policy}
% Set Pagestyle to Fancy
\pagestyle{fancy}

% Clear all header and footer fields
\fancyhf{}

% Page number only in the center of the footer
\fancyfoot[C]{\thepage}

% Remove header line
\renewcommand{\headrulewidth}{0pt}
\renewcommand{\footrulewidth}{0pt}

%ONLY EDIT BELOW, REMOVE LIPSUM (PLACEHOLDER TEXT) BEFORE EDITING
\section{Purpose}
The purpose of this Disaster Recovery (DR) Policy is to establish a structured approach for Honda to prepare for, respond to, and recover from disruptive events that impact business operations, IT infrastructure, or production lines. This policy ensures the continuity of mission-critical services and protects employees, customers, assets, and reputation.
\section{Scope}

This policy applies to:

    All Honda Group business units (e.g., Honda Motor Co., American Honda, Honda R\&D, Honda Manufacturing).
    All IT systems, data centers, networks, and applications.
    Physical facilities, including manufacturing plants, warehouses, and offices.

Employees, third-party vendors, and suppliers involved in ope
\section{Policy Statement}

Honda shall maintain a robust, enterprise-wide disaster recovery strategy that includes:

Business Impact Analysis (BIA) to identify critical functions.

Risk assessments to assess threats (natural disasters, cyberattacks, pandemics, etc.).

Disaster recovery planning includes defined Recovery Time Objectives (RTO) and Recovery Point Objectives (RPO).

Redundant systems and geographically diverse backups to ensure recoverability.

Communication plans to inform stakeholders.

Regular testing of DR plans to verify effectiveness.

Continuous improvement based on test results and incident reviews.

\section{Roles and Responsibility}

Executive Leadership
    Approve and fund disaster recovery initiatives.
    Provide strategic oversight.

Global Risk Management
    Own and maintain the DR policy.
    Coordinate audits and risk reviews.

IT Disaster Recovery Team
    Design, implement, and maintain recovery plans.
    Monitor the health and performance of the DR system.
    Conduct regular recovery tests.

Plant Managers / Site Operations
    Develop localized recovery plans.
    Coordinate facility-specific risk mitigation.

All Employees
    Follow instructions during a disaster event.
    Complete assigned disaster response training.

\section{Disaster Categories}

Natural Disasters: Earthquakes, floods, hurricanes, and wildfires, etc.

Technological Disasters: Hardware failures, data center outages, software corruption.

Cybersecurity incidents: Ransomware, DDoS attacks, data breaches.

Man-made Disruptions: Terrorist acts, sabotage, civil unrest.

Pandemics/Epidemics: Disruptions from infectious disease outbreaks.
\section{Recovery Plans}

IT Systems
    Backup and replication using cloud and off-site storage.
    Data center failover to redundant sites.
    Restore network and system within RTOs.

Manufacturing
    Alternate production facilities for critical components.
    Supply chain contingency contracts.
    Real-time monitoring of operational status.

Communications
    Multichannel alerts (SMS, email, internal apps).
    Public relations response through designated spokespersons.
    Coordination with local and national authorities.
\section{Testing and maintenance}

DR plans must be tested annually or after major changes.

Types of testing include tabletop exercises, simulations, and full-scale failures.

All test results must be documented and used to update the DR strategy.

\section{Training and Awareness}

All employees must undergo annual disaster awareness training.

DR team members require specialized technical training.

New hires must be briefed on DR procedures during on boarding.
\section{Compliance and Auditing}

DR compliance will be evaluated annually by the Internal Audit Office.

All DR efforts must comply with:

ISO 22301 (Business Continuity)

ISO/IEC 27001 (Information Security)

Local laws and regulatory requirements.