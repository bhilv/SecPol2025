%%%%%%%%% BEGIN GRC %%%%%%%%%
\chapter{Governance, Risk, and Compliance}
% Set Pagestyle to Fancy
\pagestyle{fancy}

% Clear all header and footer fields
\fancyhf{}

% Page number only in the center of the footer
\fancyfoot[C]{\thepage}

% Remove header line
\renewcommand{\headrulewidth}{0pt}
\renewcommand{\footrulewidth}{0pt}



\section{Purpose}
The purpose of the Governance, Risk, and Compliance (GRC) program is to establish a structured, integrated approach to managing these activities at Honda's Greensburg, Indiana manufacturing plant. The GRC program is designed to align the plant’s business objectives with relevant legal, regulatory, and internal requirements while effectively managing risk. By coordinating governance (policies and oversight), risk management, and compliance efforts, Honda ensures that its operations remain efficient, secure, ethically and legally sound.

\section{Scope}
The scope of Honda's GRC program covers all major operational areas of the Indiana manufacturing facility, including production processes, information systems, employee activities, and interactions with suppliers and partners. It encompasses the governance policies, risk management processes, and compliance obligations that apply to these areas. This chapter addresses GRC practices across a range of domains, including data security \& privacy, quality control, and occupational safety to environmental protection and supply chain oversight. Defining this scope ensures that GRC efforts remain focused on relevant areas and do not become overly broad or unmanageable.

\section{Roles and Responsibilities}

While many employees interact and operate within the policy standards defined within this document, those held accountable for the effectiveness, maintenance, and audit of policies held within the Honda Security Policy are detailed below.

\begin{table}[h!]
\centering
\begin{tabular}{|l|p{10cm}|}
\hline
\textbf{Role} & \textbf{Responsibilities} \\
\hline
Chief Risk Officer (CRO) & Identifying, assessing, and mitigating risks that may impact Honda's financial operations and reputation.\\
\hline
Chief Information Security Officer (CISO)& In charge of all aspects of security within Honda, including data protection, cybersecurity, and IT security.\\
\hline
Compliance Officer & Ensures that Honda adheres to all applicable laws, regulations, and industry standards.\\
\hline
Security Analyst & Responsible for the technical aspects of cybersecurity within Honda, including network monitoring and threat detection.\\
\hline
\end{tabular}
\caption{Roles and Responsibilities in Risk and Security Management}
\end{table}


\section{Establishing Policy}
Effective governance requires a strong foundation of policies, ethics, and leadership oversight. Honda ensures that:
\begin{enumerate}
    \item \textbf{Policies and Ethical Standards:} Honda will define formal policies that articulate the rules of conduct and ethical standards expected of all employees and partners. These policies create an accountability framework, clearly outlining acceptable behavior and practices, and instill a culture of integrity throughout the organization.
    \item \textbf{Strategic Alignment:} All GRC activities and policies are aligned with Honda's overall business objectives and long-term goals. This alignment ensures that risk management and compliance efforts support (and do not hinder) the plant’s operational performance and strategic direction. GRC initiatives are evaluated in the context of how they help achieve Honda’s mission and maintain its reputation.
    \item \textbf{Leadership Oversight:} Senior management at Honda’s Indiana plant is actively involved in reviewing and guiding GRC efforts. Leadership oversight provides accountability and demonstrates top-down commitment to governance and risk awareness. A governance structure (such as a GRC committee or designated executives) oversees the implementation of policies and controls, ensuring that GRC remains a priority and that issues are addressed promptly.
    \item \textbf{Roles and Responsibilities:} Clear roles and responsibilities for GRC activities are established. Honda assigns specific responsibility for areas such as compliance monitoring, risk assessment, and policy enforcement to appropriate roles (a compliance officer, risk manager, IT security lead, environmental health \& safety coordinator, etc.). By delineating who is accountable for each aspect of governance, risk, and compliance, the company ensures effective execution and avoids gaps or overlaps in coverage.
    \item \textbf{Annual Review \& Renewal} Honda will review and renew it's Incident Response, Business Continuity, Disaster Response, and Security Policy annually. This process ensures that Honda is able to adapt to current threats, regulations, and allow for the implementation of the latest established best-practices.
\end{enumerate}

\section{Risk Management}
Managing risk is a continuous, proactive process at the plant, involving identification of risks, implementation of mitigations, and ongoing monitoring:
\begin{enumerate}
    \item \textbf{Risk Identification:} Honda regularly conducts risk assessments to identify potential risks that could hinder the plant's objectives. These include operational risks (such as equipment failures or production downtime), safety hazards (workplace accidents or injuries), supply chain disruptions (late deliveries, quality issues with suppliers, or single-source dependencies), financial risks (cost fluctuations, budget overruns, or market changes), and cybersecurity threats (malware infections, data breaches, or system outages). For further information, refer to the Business Continuity Policy.
    \item \textbf{Risk Mitigation Strategies:} For each significant risk identified, Honda develops and implements plans to reduce or eliminate that risk. Mitigation strategies might include engineering solutions and preventive maintenance to address operational and safety risks, establishing backup suppliers or increasing inventory for critical materials to handle supply chain risks, purchasing insurance or setting aside contingency funds for financial risks, and deploying cybersecurity measures (firewalls, antivirus, access controls, etc.) to combat digital threats. Each mitigation plan is documented and assigned to responsible owners. For further information, please reference the Incident Response Policy.
    \item \textbf{Continuous Monitoring and Evaluation:} Risk management efforts are continuously monitored and evaluated. Honda will track key risk indicators and incident reports to gauge whether risk levels are increasing or if mitigation measures are effective. Periodic reviews of the risk register are conducted, adjusting risk mitigation strategies as needed (for example, if a new risk emerges or if existing controls are not sufficient). This ongoing evaluation helps the plant adapt to changing conditions and ensures that the risk management process remains dynamic and responsive. For further information, please reference the Business Continuity Policy.
\end{enumerate}

\section{Data Classification}
Honda will implement a data classification scheme to categorize information assets based on sensitivity and importance. By defining clear classification levels and criteria, the company can ensure that each type of data is handled with appropriate security controls and access restrictions. All data assets, both structured (such as databases, production schedules, and customer or supplier records) and unstructured (such as emails, design documents, and reports), will be inventoried and classified under this scheme.

\begin{table}[h!]
\centering
\begin{tabular}{|l|p{10cm}|}
\hline
\textbf{Classification Level}& \textbf{Definitions \& Examples}\\
\hline
Public& Information approved for public release. Disclosure of public data poses no risk
to Honda. \textit{Examples: Press releases, publicly available company information,
marketing materials.}\\
\hline
Internal& Non-public information intended for internal use within Honda. Typically of
low sensitivity, with limited impact if disclosed outside the company. \textit{Examples:
Internal memos, routine operational reports, organizational charts and phone di-
rectories.}\\
\hline
Restricted& Sensitive information that should be restricted to specific groups or departments.
Unauthorized disclosure could have a significant negative impact on the business
or competitive position. \textit{Examples: Supplier contracts and pricing, detailed pro-
duction process documentation, non-public technical specifications, project plans.}\\
\hline
Confidential& Highly sensitive information with strict access controls, limited to only those who
absolutely need it. Unauthorized disclosure of confidential data could cause severe
financial, legal, or reputational damage. \textit{Examples: Trade secrets, proprietary
research and development data, critical product designs, personally identifiable
information of employees or customers}\\
\hline
\end{tabular}
\caption{Data Classification Matrix}
\end{table}

Appropriate handling requirements (such as encryption, access control, or audit logging) will be specified for each classification level. The data classification policy helps employees understand how to treat information and prevents both accidental and malicious data leaks by ensuring higher-sensitivity data receives stronger protections.

\section{Data Retention}
Honda will retain customer, internal, and partner data for no longer than a period of seven years. While there is currently no applicable laws governing the retention of data in Indiana, federal and global legal regulations are constantly changing, and in anticipation of future changes; Honda reserves the right and obligation to store data for legal purposes. This includes but is not limited to:
\begin{enumerate}
    \item \textbf{Customer Data} is considered to be any data related to the transfer of information between Honda and it's customers. This can include personal information, including potentially personally identifiable information.
    \item \textbf{Financial Data} is considered to be any data related to the transfer of money between Honda and it's partners, customers, and suppliers. This can potentially consist of restricted or confidential information that must be handled differently than publicly available information.
    \item \textbf{Environmental Data} is considered to be any data that relates to any information gathered about Honda's effect on the environment, including markers like pH levels, chemical contaminants, spills, and carbon dioxide emissions or potentially toxic fumes.
    \item \textbf{OHS Data} is considered to be any data that is collected in relation to the health and well-being of Honda's employees. This may include information such as injury rates, safety training effectiveness, etc.
    \item \textbf{Network Data} is considered to be any data that is generated from activity on Honda's network, most commonly network logs collected from employee or corporate devices.
\end{enumerate}

\section{Data Destruction}
All applicable data no longer in retention or in-use by Honda is applicable for destruction. After retained data no longer needs to be held, it will be manually destroyed by the Security Team. The Data Destruction Policy must be followed to maintain confidentiality of sensitive data; including but not limited to cryptographic erasure, degaussing, and/or physical destruction of storage technology where applicable.

\section{Quality Management}
Quality management ensures that Honda's products meet strict standards and customer expectations. The Indiana plant employs several key quality control practices:
\begin{enumerate}
    \item \textbf{Incoming Raw Material Inspection:} All incoming raw materials and components are subject to stringent quality checks before they enter production. Inspection of materials will be determined by state guidelines, following department of transportation recommendations, (namely, the Manual For Frequency of Sampling and Testing and Basis For Use of Materials)By inspecting materials upon arrival (for example, verifying the specifications and quality of steel, plastic parts, electronics, etc.), Honda prevents defective or substandard inputs from causing problems later in the manufacturing process. 
    \item \textbf{In-Process Inspections:} Quality inspections are carried out at various stages of the manufacturing process. This might include checking critical dimensions, tolerances, and assembly steps on the production line at defined checkpoints. Regular in-process inspection helps identify any deviations from quality standards early, so that issues can be corrected immediately---reducing waste, rework, or potential recalls.
    \item \textbf{Document Control:} A robust document control system is implemented to manage all quality-related documents. This system governs the creation, review, revision, and archiving of documents such as standard operating procedures (SOPs), work instructions, quality manuals, inspection forms, and records of changes. Maintaining strict document control ensures that everyone is working off the correct, most up-to-date procedures and that there is traceability for any changes made (which is critical for both quality consistency and compliance audits).
\end{enumerate}

Additionally, Honda’s quality management system is aligned with recognized industry standards (such as ISO~9001 and the automotive-specific IATF~16949). By following these structured quality management frameworks, the plant strives for continuous improvement in processes and products, defect reduction, and high customer satisfaction.

\section{Occupational Health \& Safety}
The health and safety of employees are top priorities in Honda's operations. The plant’s Occupational Health and Safety (OHS) program includes:
\begin{enumerate}
    \item \textbf{Leadership Commitment:} Honda’s leadership at the Indiana plant demonstrates strong commitment to workplace health and safety. Management establishes clear OHS policies and objectives, and allocates the necessary resources to support safety initiatives. Leaders also lead by example in following safety rules and procedures. A culture of safety is promoted from the top down, with management encouraging active worker involvement---such as through joint management-worker safety committees---so that employees participate in hazard identification, safety discussions, and program evaluations.
    \item \textbf{Hazard Identification and Risk Assessment:} Formal processes are in place to continually identify hazards in the workplace and assess their associated risks. This includes routine inspections and job safety analyses to spot potential dangers (machine guarding issues, exposure to hazardous chemicals or fumes, ergonomic stresses in assembly tasks, electrical hazards, and common issues like slips or trips). For each identified hazard, a risk assessment is performed considering the likelihood of an incident and the potential severity of injuries or illnesses. Based on this assessment, Honda implements risk mitigation measures—ranging from engineering controls (such as installing safety guards and ventilation systems) to administrative controls (safety procedures, work rotation to reduce repetitive stress), and providing appropriate personal protective equipment (PPE)—with the aim of eliminating the hazard or minimizing the risk.
    \item \textbf{Compliance with Regulations and Standards:} The plant complies with all applicable occupational safety and health regulations. This includes adherence to OSHA standards (enforced in Indiana via the Indiana Occupational Safety and Health Administration, IOSHA) covering hazard communication (ensuring employees know about chemical hazards), emergency action and fire prevention plans, machine safeguarding and lockout/tagout procedures for equipment maintenance, use of PPE, fall protection requirements, and more. Honda also looks to industry best practices and standards for guidance; for example, the company may implement an occupational health and safety management system in line with ISO~45001 to provide a structured framework for managing OHS risks. Any other relevant national or international safety guidelines applicable to the manufacturing environment are identified and followed.
    \item \textbf{Safety Programs and Training:} Comprehensive safety programs and training ensure that employees are well-informed and prepared to work safely. All employees receive training on general workplace safety and on specific job hazards before starting work, and refresher trainings are conducted annually. Training covers proper operation of machinery and tools, safe handling of hazardous materials, emergency response procedures (such as what to do in case of a fire or chemical spill), and correct use of PPE for tasks that require it. By investing in ongoing education and drills, Honda ensures that safety procedures are understood, remembered, and practiced consistently on the factory floor. Training is to be done semi-annually for employees in low-risk environments, and bi-monthly for employees in high-risk environments. Employees who fail to meet training expectations and/or safety evaluations will be suspended and given remedial training or be reviewed for termination.
    \item \textbf{Incident Management and Continuous Improvement:} Honda has clear procedures for incident reporting and investigation. Employees are encouraged and required to report any workplace incidents, accidents, or near-misses immediately. Each report is investigated to determine root causes, and corrective actions are implemented to prevent similar incidents in the future. The plant tracks safety performance indicators (like injury frequency rates, near-miss counts, audit findings) and regularly reviews the effectiveness of its OHS programs. Management conducts periodic safety meetings and program reviews, using these insights to drive continual improvement. Through this proactive incident management and feedback process, Honda aims to continuously improve its OHS performance and move toward the goal of zero workplace injuries or illnesses.
\end{enumerate}

\section{Environmental Management}
Honda is committed to environmental responsibility and compliance in its manufacturing operations. The plant’s environmental management program focuses on:
\begin{enumerate}
    \item \textbf{Environmental Compliance:} The facility will comply with all applicable U.S. federal and Indiana state environmental laws and regulations. Key requirements include the Clean Air Act (ensuring emissions from manufacturing processes, such as paint booths or boilers, meet air quality standards and obtaining any necessary air permits through the Indiana Department of Environmental Management (IDEM)) and the Clean Water Act (managing wastewater discharges or stormwater runoff under the appropriate permits to protect waterways). The plant will also properly manage hazardous wastes in compliance with the Resource Conservation and Recovery Act (RCRA) and adhere to the Toxic Substances Control Act (TSCA) for any chemical substances used in production. Furthermore, Honda meets the requirements of the Emergency Planning and Community Right-to-Know Act (EPCRA) by maintaining up-to-date records of hazardous chemicals on-site and reporting as required to local emergency planners and responders. All environmental permits and records are maintained meticulously, and compliance inspections or audits are welcomed as opportunities to verify and improve adherence. Honda will review it's environmental compliance standards monthly, and after every applicable incident concerning environmental factors (such as water quality, air quality, or threats to public health).
    \item \textbf{Environmental Risk Management:} Beyond basic compliance, Honda proactively identifies and manages environmental risks associated with its operations as per the Disaster Recovery Policy. This includes assessing risks of spills, accidental releases of pollutants, excessive air emissions, or other environmental incidents. The plant has controls and contingency plans in place to prevent and respond to such events—for example, spill containment systems and response kits for oil or chemical leaks, emission control devices and routine maintenance to prevent air pollution exceedances, and fail-safes in wastewater treatment systems to avoid unauthorized discharges. Monthly environmental audits and risk assessments are conducted to evaluate potential weak points in environmental controls. By managing these risks, Honda protects the environment and reduces the likelihood of regulatory violations or community impacts.
    \item \textbf{Data Management and Reporting:} Honda tracks environmental performance data and complies with all required environmental reporting. This includes monitoring air emissions (including Carbon Dioxide, levels of volatile organic compounds from paint operations or other regulated pollutants) and water discharge quality on a continuous or periodic basis. The plant maintains detailed records of waste generation, chemical inventory and usage, and any environmental incidents. Honda submits required reports such as annual emissions inventories, Toxic Release Inventory (TRI) data if applicable, Tier II hazardous chemical inventory reports under EPCRA, and discharge monitoring reports for wastewater permits. Accurate data management and timely reporting ensure transparency with regulators and the public, and they help the company measure progress toward environmental goals.
    \item \textbf{Technological Support:} The plant leverages technology to support environmental compliance and performance. For example, continuous emissions monitoring systems (CEMS) will be used on specified equipment (as per the Environmental Protection Policy) to provide real-time data on air emissions. Sensors and automation in wastewater treatment can monitor pH, chemical levels, or flow rates and alert staff to any out-of-range conditions. Environmental management software may be used to track permit requirements, due dates for inspections or trainings, and to store documentation (in the form of safety data sheets, inspection records, and regular auditing). By using modern technology and data analytics, Honda can more effectively identify trends (such as a gradual increase in energy or water usage) and target opportunities to reduce environmental impact. Technology also helps in early detection of potential compliance issues so that they can be corrected before becoming problems.
\end{enumerate}

\section{Supply Chain Management}
A resilient and compliant supply chain is critical to uninterrupted production at the Honda plant. Honda’s supply chain management efforts include:
\begin{enumerate}
    \item \textbf{Supply Chain Risk Assessment:} The company identifies and assesses risks throughout its supply chain that could disrupt production or impact compliance. This involves evaluating suppliers and logistics for potential points of failure. Examples of risks include over-reliance on single-source suppliers for key components, suppliers located in regions prone to natural disasters or political instability, quality control issues at a vendor that could lead to defective parts, or even cybersecurity vulnerabilities at a supplier that could affect Honda’s systems (via compromised parts or data exchange). Each supplier or material is reviewed for risks such as late deliveries, capacity issues, or financial instability. Refer to the Business Continuity Plan for more information.
    \item \textbf{Risk Mitigation and Continuous Monitoring:} For significant supply chain risks identified, Honda implements mitigation strategies and continuously monitors the situation. Mitigation can include qualifying multiple suppliers for critical parts (to avoid single points of failure), maintaining safety stock or buffer inventory for components with long lead times, and working closely with suppliers on quality improvement programs. Honda also stays vigilant by monitoring supplier performance metrics and external indicators (like market reports or news that might signal a supplier problem). Through regular communication with key suppliers and the use of supplier management tools, the plant keeps an eye on supply chain health in real time. This way, if a potential disruption is looming (for example, a supplier hinting at capacity issues or financial troubles), Honda can proactively adjust its plans or source alternatives.
    \item \textbf{Regulatory Compliance in Sourcing:} Honda will identify and adhere to any regulatory requirements that apply to its sourcing and materials. For instance, if the manufacturing process involves materials that fall under specific legal regulations (such as conflict minerals like tin, tungsten, tantalum, and gold, which require due diligence and reporting under U.S. law), the company ensures those requirements are met. Additionally, import/export regulations, trade compliance (including tariffs or restricted trade partner screening), and environmental or safety regulations related to materials (like restrictions on hazardous substances in parts) are all considered when selecting suppliers and materials. By ensuring that sourced parts and materials comply with relevant laws and standards, Honda avoids legal complications and upholds ethical sourcing principles.
    \item \textbf{Supplier Compliance and Internal Controls:} The company conducts due diligence to ensure suppliers meet Honda’s standards and comply with relevant laws. This may include requiring suppliers to sign codes of conduct or contractual clauses affirming compliance with labor laws, environmental regulations, quality standards, and cybersecurity practices. Honda might request audits or certifications from high-risk suppliers (for example, verifying that a supplier’s facility meets ISO~9001 quality management standards or that they maintain proper cybersecurity controls if they handle Honda’s data). Internally, Honda maintains controls such as approved supplier lists (only doing business with vetted suppliers), regular supplier performance reviews, and a process for addressing any supplier non-compliance or incidents. If a supplier is found to violate critical compliance requirements or to pose excessive risk, Honda will take corrective action, which could include helping the supplier improve or phasing out that supplier. These internal controls and supplier management practices ensure that the supply chain remains robust, ethical, and aligned with Honda’s compliance obligations.
\end{enumerate}

\section{Data Privacy \& Security}
Protecting sensitive data and maintaining robust cybersecurity is vital for Honda’s operations. Key elements of the plant’s data privacy and security strategy include:
\begin{enumerate}
    \item \textbf{Sensitive Data Identification and Mapping:} Honda identifies all forms of sensitive data that the plant handles. This includes intellectual property (such as vehicle designs or proprietary manufacturing processes), confidential business information (production volumes, pricing, strategic plans), personal data of employees (HR records, health information) or customers, and any other critical information assets. The flow of this data through the organization is mapped out---detailing where data is collected, how it is stored and processed, and where it is transmitted or shared (including with external partners or Honda headquarters). Understanding data flows and storage locations allows the company to pinpoint vulnerability points and apply appropriate safeguards at each step.
    \item \textbf{Privacy and Data Protection Compliance:} Honda ensures compliance with all applicable data privacy laws and regulations. While the Indiana plant primarily operates under U.S. law, Honda takes into account relevant federal and state regulations. For example, if any personal information of consumers is collected (such as customer data for vehicle telematics or marketing), Honda will comply with the California Consumer Privacy Act (CCPA) regarding notice, data use, and honoring consumer rights for California residents. If the business involves handling financial customer data (for instance, through any vehicle financing programs or credit applications), the plant will follow the data safeguard requirements of the Gramm-Leach-Bliley Act (GLBA). Additionally, Indiana’s data breach notification law is adhered to: in the event of a data breach involving personal information, Honda must notify affected Indiana residents and the state Attorney General within seventy-two hours. On a broader scale, Honda is mindful of international standards such as the EU’s General Data Protection Regulation (GDPR) when dealing with any global data to ensure proper consent, data handling, and cross-border transfer practices. Semi-annual internal audits and reviews are conducted to verify that data is being handled in accordance with these laws and Honda’s own privacy policies.
    \item \textbf{Security Framework and Standards:} Honda’s cybersecurity program at the plant is built on industry best practices and frameworks. The company aligns its security controls with the National Institute of Standards and Technology (NIST) Cybersecurity Framework and incorporates standards from ISO/IEC~27001 for information security management. These frameworks guide the plant in covering all aspects of cybersecurity: identifying assets and risks, protecting systems with appropriate safeguards, detecting security events, having incident response plans, and establishing recovery plans for continuity. Furthermore, if any operations involve U.S. government data or contracts, Honda will implement the required security controls (for example, complying with NIST~SP~800-171 for protecting controlled unclassified information in government-related projects). Embracing these recognized frameworks provides a comprehensive and systematic approach to managing cybersecurity risks.
    \item \textbf{Zero Trust Architecture:} The plant employs a Zero Trust security architecture. In practice, this means that no user or device is inherently trusted, even if it is within the corporate network. Every access request to resources (applications, databases, network segments) is continuously verified; Users must authenticate with multi-factor authentication when accessing Restricted or Confidential information and be authorized for the specific action or data each time. Network segmentation is used to isolate sensitive systems, and additional verification (such as device security posture checks) is required before granting access. By assuming that threats can exist both inside and outside the traditional network perimeter, the Zero Trust approach greatly reduces the risk of an insider threat or a compromised account moving laterally across systems.
    \item \textbf{Access Controls and Least Privilege:} Honda enforces strict access control measures to ensure that employees and systems only have the minimum access necessary to perform their duties. This principle of least privilege is applied to user accounts, system accounts, and even application permissions. Access to sensitive systems (like financial data, confidential design documents, or critical production controls) is limited to authorized personnel based on role, and those access rights are reviewed regularly. Strong authentication mechanisms (including passwords that meet complexity requirements and multi-factor authentication as per the Password Policy) are in place to prevent unauthorized access. When employees change roles or leave the company, their access rights are promptly adjusted or revoked as part of an off-boarding procedure. An Audit of all current access controls will be done semi-annually for all user and system accounts.
    \item \textbf{Data Encryption:} All sensitive data is protected through encryption both in transit and at rest. For data in transit, Honda uses secure communication protocols such as HTTPS/TLS for web traffic, secure shell (SSH) for remote admin access, and VPN tunnels for any remote connections to the plant’s network. This prevents eavesdropping or interception of data as it flows between systems. For data at rest, technologies like full-disk encryption on laptops, encryption and tokenization of databases and file systems on servers, and encrypted backup media are employed. This means that if a device or storage media were to be lost or stolen, the data would remain unreadable and protected from unauthorized access. More information can be found in the Encryption Policy.
    \item \textbf{Data Loss Prevention:} To prevent the unauthorized leakage of sensitive information, Honda utilizes Data Loss Prevention (DLP) measures. These include software tools that monitor and control the transfer of data through various channels (such as email, USB drives, or web uploads). If, for example, an employee attempts to send out a file containing confidential plans or export a large amount of production data, the DLP system can detect this based on content scanning and either block the action or flag it for the security team’s review. DLP policies are configured to balance security with business needs, ensuring that legitimate data sharing is allowed while risky or non-compliant transfers are stopped.
    \item \textbf{Secure Storage and Communications:} The plant uses secure methods for storing and sharing information. Important files and intellectual property are stored in access-controlled repositories or document management systems that track versions and access history. When sharing sensitive data with suppliers or partners, Honda employs secure file transfer solutions or encrypted email rather than sending data over open channels. Within the manufacturing facility, any network-connected equipment or Internet of Things (IoT) devices are secured to prevent them from becoming entry points; this may include using segregated networks for certain equipment and ensuring firmware is kept updated. Overall, communications that involve sensitive or proprietary information are protected so that eavesdropping or tampering is prevented.
    \item \textbf{Employee Training and Awareness:} Employee awareness is a critical component of data security. Honda provides regular training to employees about cybersecurity best practices and data protection. This includes educating staff on how to recognize phishing emails or social engineering attempts, proper use of company devices, secure password creation and management, and the importance of reporting suspicious incidents promptly. Specialized training is given to those in high-risk roles (for instance, IT administrators or those handling sensitive data) to ensure they understand the specific threats and responsibilities they have. By fostering a culture of security awareness, the plant reduces the likelihood of accidental security breaches and empowers employees to act as an additional line of defense. For further information, refer to the Security Awareness Policy.
    \item \textbf{Incident Response and Recovery Plans:} Despite all preventive measures, Honda prepares for the possibility of cybersecurity incidents or data breaches: therefor, Honda has mandated the creation of a Business Continuity, Disaster Recovery, and Incident Response Plans. A documented Incident Response Plan (IRP) is in place, which outlines the steps to be taken in the event of a security incident---such as a malware outbreak or detected data breach. The IRP defines roles (such as who is on the response team), communication channels (how to escalate issues to management and, if needed, to law enforcement or affected parties), and procedures for containment, eradication of threats, recovery of systems, and post-incident analysis. In conjunction, a Disaster Recovery Plan (DRP) exists to address how the plant would restore critical operations and data in case of a major disruptive event (cyber-related or even physical disasters). This includes regular backups of key systems and data, off-site storage of backups, and drills or simulations to test recovery times. Together, these plans ensure that Honda can respond swiftly to incidents, minimize damage, and recover normal operations as quickly as possible. For more information, refer to the Incident Response, Disaster Recovery, and Business Continuity Plans.
    \item \textbf{Security Audits and Vulnerability Management:} The plant undergoes semi-annual security audits and continuous vulnerability management. Security audits (internal and external) review the effectiveness of controls and compliance with policies. For example, an audit might check that firewall rules are properly set, user accounts are managed correctly, or that security cameras and access controls are functioning in restricted areas of the facility. On the technical side, Honda employs vulnerability scanning tools that routinely scan the network, servers, and applications for known vulnerabilities or misconfigurations. Critical systems are also subject to annual penetration testing to simulate attacks and uncover any weaknesses that scanners might not catch. When vulnerabilities are identified, security procedures and patches will be updated or applied as per the vulnerability's applicable policy.
    \item \textbf{Monitoring and User Activity Logging:} Continuous monitoring is a cornerstone of Honda’s security posture. The plant uses security information and event management, including but not limited to: (SIEM) systems to aggregate and analyze logs from various sources, network devices, servers, workstations, and security appliances. Alerts are configured to notify the security team of unusual patterns (such as repeated failed logins, after-hours access to sensitive systems, or unrecognized devices connecting to the network). Critical systems and areas are monitored via surveillance and alarms to detect any physical intrusion or unauthorized access. Additionally, user activity on sensitive systems is logged and continuously monitored to ensure that employees are adhering to policies and not engaging in risky behavior. This monitoring respects privacy laws and focuses on detecting genuine threats or misuse. By maintaining vigilant monitoring and timely review of security events, Honda can quickly detect, investigate, and respond to potential security issues before they escalate.
\end{enumerate}

%%% END GRC %%%