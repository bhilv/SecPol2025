\documentclass{article}

\title{Encryption}
\author{Elijah Quiambao}
\date{July 2025}

\begin{document}
\maketitle

\section{Purpose}
The purpose of this policy is to provide guidance that limits the use of encryption to those algorithms that have received substantial public review and have been proven to work effectively. Additionally, this policy provides direction to ensure that Federal regulations are followed, and legal authority is granted for the dissemination and use of encryption technologies outside of the United States. The purpose of this document is also also protect customer data and to ensure confidentiality with all Honda's sensitive data. 

\section{Scope}
This policy's applies to all Honda employees and affiliates.

\section{Safeguards}
Ciphers in use must meet or exceed the set defined as "AES-compatible" or "partially AES-compatible" according to the IETF/IRTF Cipher Catalog, or the set defined for use in the United States National Institute of Standards and Technology (NIST) publication FIPS 140-2, or any superseding documents according to the date of implementation. The use of the Advanced Encryption Standard (AES) is strongly recommended for symmetric encryption.
Algorithms in use must meet the standards defined for use in NIST publication FIPS 140-2 or any superseding document, according to date of implementation. The use of RSA and Elliptic Curve Cryptography (ECC) algorithms is strongly recommended for asymmetric encryption. These safeguards ensure to keep customer privacy confidential. All sensitive vehicle data such as Telematics and CAN should be encrypted using authenticated encryption (AES-GCM) to prevent replay attacks in accordance with honda's vehicle security standards. Telematics is a system in which a vehicle sends and receives data and is used for GPS tracking. Telematics ensures that vehicles can communicate with external systems. All Third party partners are required to adhere to the

\end{document}

\